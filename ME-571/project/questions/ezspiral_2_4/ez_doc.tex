%-----------------------------------------------------------------------------%
%-----------------------------------------------------------------------------%
%                                                                             %
%                                 EZ-SPIRAL                                   %
%                     A Code for Simulating Spiral Waves                      %
%                                                                             % 
%                                 Version 2                                   %
%                                                                             %
%                                                                             %
% $Revision: 2.4 $
% $Date: 1997/11/14 22:32:24 $
% 
%-----------------------------------------------------------------------------%
%-----------------------------------------------------------------------------%

% This is a LaTex source file.

\documentstyle[12pt]{article}
\textheight 9in  \textwidth 6.5in
\topmargin -.7in \oddsidemargin .0in

\begin{document}
\baselineskip = 20pt plus 1pt minus 1pt
\parskip = 8pt
\parindent=0pt

\begin{center} 
{\bf EZ-SPIRAL DOCUMENTATION} 
\end{center} 

\bigskip
{\bf I. Philosophy of EZ-Spiral} 
\smallskip

EZ-Spiral is a simplified version of the C program that I have used to study
the dynamics of spiral waves in excitable media [1-4,6]. EZ-Spiral is designed 
to be as simple as possible while offering what I believe to be the 
most important features for spiral simulations. These include:
\begin{list}{dummy}
{\labelwidth=0.2in\labelsep=4pt\partopsep=-10pt\parsep=-10pt\itemsep=12pt}
\item[(1)] Continuous range of spatio-temporal resolutions.
\item[(2)] Choice of implicit or explicit time stepping of the reaction 
           kinetics. 
\item[(3)] Choice of ratio of diffusion coefficients.
\item[(4)] Computation of spiral-tip location.
\item[(5)] Automatic generation of initial conditions.
\item[(6)] Interactive graphics with control of spiral location.
\end{list}

The code should be of use to those who are interested in the serious
study of excitable media as well as those who simply want to generate
pretty pictures. Most of the documentation is contained in the code
itself. The notation for the model reaction-diffusion equations and a
description of parameters are given in Sec. IV.

{\bf Changes in Version 2.}  Version 2 of EZ-Spiral fixes many of the
awkward properties of Version 1.  In particular, almost all parameters
one routinely wants to change are now contained in {\em task.dat} and
are read in at run time.  This eliminates most of the recompiling that
was necessary with Version 1.  The other major change is that the
graphics subroutines have been rewritten using X11 calls and hence the
code is portable to a large number of machines.  Elementary graphics
are even available for black and white X-terminals.  Finally, parts of
the code have been cleaned up and simplified.  By editing just two
lines it is now possible to simulate a large variety of models with
EZ-Spiral (see Sec. IV).  I include at the end (Sec. V) a simple
recipe for obtaining turbulent spirals.  Note that the format of {\em
task.dat} has changed slightly from Version 1.  The format of the
field files has not changed.

{\bf Changes in Version 2.3} The primary change in Version 2.3 is that
the the higher-order implicit $u$ time stepping is implemented.  See [6].

\pagebreak

\medskip
{\bf II. Running EZ-Spiral} 
\smallskip

{\bf Files:} Be certain that you have the following files: \\ {\em
ezspiral.c, ezstep.c, eztip.c, ezgraph.c, ezspiral.h, ezstep.h
ezgraph.h, task.dat}, and {\em Makefile}. \\ You will probably want to
save copies of these files (in compressed tar format).

{\bf make}: It is up to you to edit {\em Makefile} as necessary for
your system. All you need is an ANSI C compiler that is smart enough
to do elementary optimizations. I recommend gcc. Typing {\em make}
will make {\em ezspiral}.  If you edit one of the source or header
files you must remake EZ-Spiral.

{\bf Running on an X-terminal:} All parameters are initially set to
run on an X-terminal. Check that {\sf GRAPHICS} is defined to be 1 at
the top of {\em ezspiral.h}. Make {\em ezspiral} by typing {\em
make}. Then run by typing {\em ezspiral}. A window should open and a
spiral should be generated.  The $v$-field is plotted. This is a
coarse resolution (CA-type) run showing the speed possible with
EZ-Spiral simulations.  With the pointer in the EZ-Spiral window, you
can move the spiral around using the arrow keys on the keyboard.
(Should the program not respond to key presses when the pointer is in
the spiral window, set {\sf FOCUS\_CK} to 1 in {\em ezgraph.h}.  If
you are running on a SUN workstation will probably have to do this.
You can read the documentation in {\em ezgraph.h} about this.)  Center
the spiral in the box.

A simulation will finish when either:

\begin{list}{dummy}
{\labelwidth=0.1in\labelsep=0pt\partopsep=-10pt\parsep=-10pt\itemsep=0pt}
\item[~]the number of time steps set in {\em task.dat} is taken, \\
	or you type {\em q, Q}, or {\em ESC} with the pointer in the spiral 
        window, \\
	or you use the menu produced by the window manager (assuming there
        is one) to kill the graphics window.
\end{list}

Typing either {\em p} or {\em P} within the spiral window will pause
the simulation until you type anything else on the keyboard (or until
you move the pointer into the spiral window if {\sf FOCUS\_CK} is set
to 1).

After a successful run, you will have a file {\em fc.dat} in your
directory which contains the final conditions of the run. If you copy
this file to {\em ic.dat}, then the next time you run {\em ezspiral},
this file will be read and used as an initial condition.

To get some insight into what the code can do, try the following runs
in order:
\begin{list}{dummy}
{\labelwidth=0.2in\labelsep=4pt\partopsep=-10pt\parsep=-10pt\itemsep=12pt}

\item[(1)] 
In {\em task.dat}, change {\sf ndim} from 51 to 81, change {\sf Time
steps per plot} to 16, and change {\sf verbose} from 2 to 1 (the normal
value). Copy {\em fc.dat} from the run with {\sf ndim}=51 to {\em
ic.dat}. Run {\em ezspiral}.  This is a more accurate run than with
{\sf ndim}=51.

\item[(2)] 
Copy {\em fc.dat} from previous run to {\em ic.dat}, change change {\sf L}
from 40 to 30, and run {\em ezspiral} again. Center the spiral. When
this run finishes, again copy {\em fc.dat} to {\em ic.dat}. Now change
change {\sf L} from 30 to 24, and run {\em ezspiral}. Again center the
spiral.  This is getting close to a fully resolved simulation of the
underlying PDEs.

\item[(3)] 
Change {\sf Time steps per locating spiral tip} in {\em task.dat} from
0 [i.e.~no tip finding] to 1 [i.e.~finding tip every time
step]. Change {\sf Time steps per write to data files} from 0 to
10. Change the $(i,j)$ {\sf history point} from $(0,0)$ to
$(10,12)$. Move {\em fc.dat} from previous run to {\em ic.dat} and run
{\em ezspiral}. You will now see the path of the spiral tip plotted
during the simulation. The tip path will be a circle.  After this run
you will have a file {\em path.dat} with the path of the spiral tip
$(t, x_{tip}, y_{tip})$, and a file {\em history.dat} with the time 
series of values $(t,u,v)$
at the grid point $(10,12)$. {\em Note that for coarse simulations,
tip finding can produce uncertain results and will probably fail.}

\item[(4)] 
Change {\sf u\_plot} in {\em task.dat} from 0 to 1, and re-run {\em
ezspiral}. Both the $u$ and $v$ fields will now be plotted. [For the
$u$-field, blue denotes quiescent, red denotes excited, and black
denotes the reaction-zone between the two $(0.1<u<0.9)$.]  

\item[(5)] 
Change the kinetics parameter {\sf a} in {\em task.dat} from 0.75 to
0.62 and {\sf Number of time steps to take} from 1000 to 2000.
Run again. With these parameters the spiral will ``meander''.
Don't panic, it probably will not meander off the screen.

\end{list}
You should have the hang of it by now.

{\bf III. Further Comments on EZ-Spiral} 

It is expected that you will modify the code as needed for your
purposes.  The code should be more or less self-explanatory. The bulk
of the code is devoted to I/O, graphics, and tip finding. Almost all
of the execution time is spent on just a few lines of code in the
routine {\sf step()} in {\em ezstep.c}.

Note that I have used many external (i.e.~global) variables and have
not used any structures.  I thought that this would make the code more
readable, and therefore usable, to those not entirely familiar with C.
I believe that the crucial {\sf for} loops in routine {\sf step()}
compile to very efficient code. See comments in {\em ezstep.h}.

I make no claims as to what constitutes a fully resolved simulation of
the underlying PDEs.  If you are concerned about this I would
recommend having at least 6 points across the interface (black in the
$u$-field plot) and a ratio of {\sl dt}/$\epsilon<0.5$.  You might
also decrease {\sf delta} by an order of magnitude or more.  You
should experiment for yourself.  Note that for {\sl dt}/$\epsilon<1$
you can use either the explicit or the implicit form of the reaction
kinetics.  The explicit form is slightly faster, but the new implicit
form [6] is definitely more accurate and I highly recommend it.  For
{\sf delta} very small, you should remove the {\sf if} from the {\sf
for} loops in {\sf step()}.

In general I believe in the 9-point formula for the Laplacians [1,6],
rather than the 5-point formula.  The 9-point formula maintains
rotational symmetry to leading order.  For coarse-grained simulations
with $D_v=0$ the difference in execution time is small compared with
what is gained.  If you are unconvinced, run the simulation described
in (1) with both 5-point and 9-point Laplacians.

For slightly more speed you can increase {\sf ts} (to as much as 1.0),
but this is not recommended.

I have included code for imposing periodic boundary conditions.  I
suggest using {\sf ndim} odd for Neumann BCs and {\sf ndim} even for
periodic BCs.

I believe tip finding is quite reliable for well resolved simulations.
For coarse-grained simulations it may fail miserably.

With {\sf GRAPHICS} in {\em ezspiral.h} set to 0, none of the X11
graphics is compiled and this should allow the code to run on {\em
any} machine.  However, do not run EZ-Spiral on a vector machine
without appropriate modification.

{\bf IV. Equations and Parameters} 

The model reaction-diffusion equations are [1,2]:
$$
\frac{\partial u }{ \partial t} = \nabla^2 u + 
	\epsilon^{-1} u(1-u)(u-u_{th}(v)), ~~~
\frac{\partial v }{ \partial t} = D_v \nabla^2 v + g(u,v)
$$
where in the simplest case
$$
u_{th}(v) = {{v+b}\over{a}}, ~~~
g(u,v) = u-v, 
$$ 
so that $a,b,$ and $\epsilon$ are parameters of the reaction
kinetics.  $D_v$ is the ratio of diffusion coefficients ($D_u \equiv
1$ by choice of length scales). In addition, $L$ is the length
of a side of the domain. Thus the ``physical'' parameters in the
simulation are: $a, b, \epsilon, L$, and $D_v$.

The method employed in EZ-Spiral is (essentially) independent of the
choice of the functions $u_{th}(v)$ and $g(u,v)$ and the user is free
to set these to whatever is desired.  See {\em ezstep.h}.

The ``numerical'' parameters for the simulation are: {\sf ndim}, {\sf
ts} = time step as fraction of the diffusion stability limit, and {\sf
delta} = small numerical parameter [1,2].  The other parameters set in
{\em task.dat} are:

\begin{list}{dummy}
{\labelwidth=0.1in\labelsep=0pt\partopsep=-10pt\parsep=-10pt\itemsep=0pt}
\item[~]{\sf Number of time steps to take} \\
        {\sf Time steps per plot} \\
        {\sf Time steps per locating spiral tip} \\
        {\sf Time steps per write to data files} \\
        $(i,j)$ {\sf history point} \\
	{\sf u\_plot} \\
	{\sf v\_plot} \\
	{\sf verbose} 
\end{list}

These are more or less self-explanatory. {\sf Time steps per locating
spiral tip} should generally be either 0 [no tip finding] or 1
[finding tip every time step].  For very small time steps a larger
value can be used. If the $(i,j)$ point is not $(0,0)$ and in the
domain, then a time series at the $(i,j)$ grid point will be saved
every {\sf Time steps per write} assuming it is not zero.  {\sf Time
steps per write} controls the output to both {\em path.dat} and {\em
history.dat}.

{\sf u\_plot} and {\sf v\_plot} control which fields are plotted.  If
both are set to zero, EZ-Spiral produces no graphics window.
Depending on the hardware you have, plotting can take a significant
amount of time, so choose {\sf Time steps per plot} judiciously.  Also
note that plotting the $u$-field is much faster than plotting the
$v$-field. See the code.

{\bf V. Spiral Turbulence with EZ-Spiral} 

I include in {\em ezstep.h} two choices for the slow kinetics $g(u,v)$
other than the choice $g(u,v) = u-v$.  These are $g(u,v) = u^3-v$ and
the function given by B\"ar and Eiswirth[5].  Both choices produce
spiral breakup followed by spiral ``turbulence''.  Here I give a
simple recipe for spiral turbulence.

In {\em ezstep.h} turn off the ``standard model'' by putting {\sf \#if
0} and turn on the ``simple model for turbulent spirals'' by putting
{\sf \#if 1}.  Remake ezspiral.  Then in {\em task.dat} set: {\sf
a=0.75, b=0.06, 1/eps=12.0, L=80, Dv=0, ndim=121, ts = 0.8}, and {\sf
delta=1.e-4}.  Set the graphics parameters to your choice.  Now run
ezspiral.  You should see spiral breakup followed by spiral
turbulence. This simulation is only qualitatively correct.  For a more
quantitative and convincing example, set {\sf 1/eps=14} and {\sf
ndim=241} (and all other parameters as before).  This time you will
see the formation of a single spiral.  After running sufficiently long
that you are convinced that the spiral is stable, stop the simulation
and move {\em fc.dat} to {\em ic.dat}.  Set {\sf 1/eps=12} and
rerun.  You will now see the spiral breakup.  You can increase the
spatial resolution still further if you desire and you can take
several small steps in {\sf 1/eps} if you wish, but I believe you will
get essentially the same result.


{\bf References} 

[1] D. Barkley, M. Kness, and L.S. Tuckerman, Phys. Rev. A {\bf 42}, 2489 (1990).

[2] D. Barkley, Physica {\bf 49D}, 61 (1991).

[3] D. Barkley, Phys. Rev. Lett {\bf 68}, 2090 (1992).

[4] D. Barkley, Phys. Rev. Lett. {\bf 72}, 164 (1994).

[5] M. B\"ar and  M. Eiswirth, Phys. Rev. E {\bf 48}, 1635 (1993).

[6] M. Dowel, R.M. Mantel and D. Barkley, ``Fast simulations of waves
in three-dimensional excitable media,'' Int.~J.~Bif.~Chaos (in press).


Please send comments to barkley@maths.warwick.ac.uk

\end{document} 

\documentclass[12pt]{beamer}
\usepackage[utf8]{inputenc}
\usepackage[T1]{fontenc}
\usepackage{lmodern}
 \usepackage{graphicx}
\usepackage[]{babel}
\usetheme{Madrid}
\begin{document}
	\author{Brian KYANJO \\ Boise State University}
	\title{Riemann Problem for Linear Scale Advection Equation (LSAE)}
	%\subtitle{}
	%\logo{}
	%\institute{}
	%\date{}
	%\subject{}
	%\setbeamercovered{transparent}
	%\setbeamertemplate{navigation symbols}{}
	\begin{frame}[plain]
		\maketitle
	\end{frame}
	
	\begin{frame}
		\frametitle{Contents}
		\begin{itemize}
			\item Introduction
			\item Bahaviour of a discontinuity
			\item Flux at the interface
			\item Approximations
			%\item Interpolation
			\item Applications
		\end{itemize}
	\end{frame}

\begin{frame}
	\frametitle{Introduction}
    
    	 \textbf{Riemann Problem}: is a specific intial value problem  (Cauchy problem) of a PDE that consists of conservation equations combined with piecewise constant intial data which has a single discontinuity in the domain of interest.
    \pause
    Consider a linear scalar advection equation.
    \begin{equation}
    	q_{t} + a q_{x} = 0
    \end{equation}
	 
	 \begin{equation}
	 	q(x,0) = h(x)  = \begin{cases}
	 		q_{L}, & \text{if $x \le 0,$}\\
	 		q_{R},& \text{if $x > 0,$}\\
	 		
	 	\end{cases}       
	 \end{equation}
 \pause
	where $q_{R}$ and $q_{L}$ are two piecewise constant states separated by a discontinuity.
	\end{frame}

\begin{frame}
	\frametitle{Continuation}
	Applying method of characteristics to the IVP gives  equation of the trajectory characteristic curve.
	
	$$x = x_{o} + at $$
	\pause 
	This is used to obtainthe exact solution of the Riemann problem
	\begin{equation}
		q(x,t) = h(x-at)  = \begin{cases}
			q_{L}, & \text{if $x - at \le 0,$}\\
			q_{R},& \text{if $x - at > 0,$}\\
			
		\end{cases}       
	\end{equation}
	%\pause

\end{frame}

\begin{frame}
	\frametitle{Continuation}
		% TODO: \usepackage{graphicx} required
	\begin{figure}
		\centering
		\includegraphics[width=0.7\linewidth]{"../screenshoots/Screenshot 2021-02-11 at 2.49.24 AM"}
		\caption{Sketch of the exact solution of the Riemann problem for the lSAE for $a > 0$.}
		\label{fig:exact_solution}
	\end{figure}
	
	
\end{frame}

\begin{frame}
	\frametitle{Continuation}
	Solving the IVP by changing to Lagrangian coordinates 
	$$\dfrac{dx}{dt} = a \quad \text{(ODE)} \quad x(0) = x_{o}$$
	Considering a = 1, we obtain two descriptions: Eulerian and Lagrangian
	
\end{frame}

\begin{frame}
	\frametitle{Continuation}
	\begin{figure}
		\centering
		\includegraphics[width=1\linewidth]{"../screenshoots/Screenshot 2021-02-11 at 3.07.29 AM"}
		\caption{Eulerian description (on the left) and the Lagrangian description (on the right)}
		\label{fig:screenshot-2021-02-11-at-3}
	\end{figure}
	
	
\end{frame}

\begin{frame}
	\frametitle{Bahaviour of a discontinuity}
	\textbf{Reimann Solver:} Is a numerical method used to solve a Riemann Problem.  
	
	\pause
	They are heavily  used in: 
	\begin{itemize}
		\item Computational Fluid Dynamics 
		\item Computational Magnetohydro Dynamics.
	\end{itemize}
	\pause
	Consider using Finite volume to the solution using approximate volume averages.
\end{frame}

\begin{frame}
	\frametitle{Continuation}
	\begin{figure}
		\centering
		\includegraphics[width=0.7\linewidth]{"../screenshoots/Screenshot 2021-02-11 at 11.40.36 AM"}
		\caption{Approximate of Volume averages}
		\label{fig:screenshot-2021-02-11-at-11}
	\end{figure}
	
\end{frame}

\begin{frame}
	\frametitle{Continuation}
	\begin{figure}
		\centering
		\includegraphics[width=0.7\linewidth]{"../screenshoots/Screenshot 2021-02-11 at 11.40.09 AM"}
		\caption{Flux reconstruction}
		\label{fig:screenshot-2021-02-11-at-11}
	\end{figure}
	
\end{frame}

\begin{frame}
	\frametitle{Continuation}
	The Fundamental similarity between the two cases is that, in both cases, the correct flux, is the flux after an infinestimal time.
\end{frame}
\begin{frame}
	\frametitle{Approximations}
	
	\textbf{Reiman Solvers include:}
	\begin{itemize}
		\item Linear Riemann Solver  (ROE) - Fast, medium accuracy
		\item Harten-Laxvan-Leer Solver (HLL)  - Problems with contact discontiniuities
		\item  Two-Shock Reimann Solver - Problems with entropy waves
	\end{itemize}
\end{frame}

%\begin{frame}
%	\frametitle{Interpolation}
%	\begin{itemize}
	%	\item Linear  (numerical stability)
%		\item Parabolic 
	%	\item High order
%	\end{itemize}
	
%\end{frame}

\begin{frame}
	\frametitle{Applications}
	\begin{itemize}
		\item Shock capturing
		\item Study of Heating, viscosity
	\end{itemize}
	
\end{frame}

\begin{frame}
	\frametitle{References}
	\begin{itemize}
		\item https://www.ita.uni-heidelberg.de/~dullemond/lectures/num\textunderscore fluid\textunderscore 2011/Chapter\textunderscore7.pdf
		\item Advanced Numerical Methods for Hyperbolic Equations and Applications Lecture notes by Michael Dumbser.
	\end{itemize}
	    
	    
	
	
	
\end{frame}

\end{document}